%%%%%%%%%%%%%%%%%%%%%%%%%%%%%%%%%%%%%%%%%%%%%%%%%%%%%%%%%%%%
%%  This Beamer template was created by Cameron Bracken.
%%  Anyone can freely use or modify it for any purpose
%%  without attribution.
%%
%%  Last Modified: January 9, 2009
%%

\documentclass[xcolor=x11names,compress]{beamer}

%% General document %%%%%%%%%%%%%%%%%%%%%%%%%%%%%%%%%%
\usepackage{graphicx}
\usepackage[utf8]{inputenc}
\usepackage[T1]{fontenc}
\usepackage[french]{babel}
\usepackage{amsmath,amssymb,amsthm,amsopn}
\usepackage{mathrsfs}
\usepackage{graphicx}
%\usepackage{tikz}
%\usepackage{array}
%\usepackage[top=1cm,bottom=1cm]{geometry}
%\usepackage{listings}
%\usepackage{xcolor}
\usepackage{hyperref}
\hypersetup{
    colorlinks=true,
    linkcolor=blue,
    citecolor=red,
}

\newtheoremstyle{break}%
{}{}%
{\itshape}{}%
{\bfseries}{}%  % Note that final punctuation is omitted.
{\newline}{}

\newtheoremstyle{sc}%
{}{}%
{}{}%
{\scshape}{}%  % Note that final punctuation is omitted.
{\newline}{}

\theoremstyle{break}
\newtheorem{thm}{Théorème}[section]
\newtheorem{lm}[thm]{Lemme}
\newtheorem{prop}[thm]{Proposition}
\newtheorem{cor}[thm]{Corollaire}

\theoremstyle{sc}
\newtheorem{exo}{Exercice}

\theoremstyle{definition}
\newtheorem{defi}[thm]{Définition}
\newtheorem{ex}[thm]{Exemple}

\theoremstyle{remark}
\newtheorem{rem}[thm]{Remarque}

% Raccourcis pour les opérateurs mathématiques (les espaces avant-après sont
% modifiés pour mieux rentrer dans les codes mathématiques usuels)
\DeclareMathOperator{\Ker}{Ker}
\DeclareMathOperator{\Id}{Id}
\DeclareMathOperator{\Img}{Im}
\DeclareMathOperator{\Card}{Card}
\DeclareMathOperator{\Vect}{Vect}
\DeclareMathOperator{\Tr}{Tr}
\DeclareMathOperator{\Mod}{mod}
\DeclareMathOperator{\Ord}{Ord}
\DeclareMathOperator{\ppcm}{ppcm}


% Nouvelles commandes
\newcommand{\ps}[2]{\left\langle#1,#2\right\rangle}
\newcommand{\ent}[2]{[\![#1,#2]\!]}
\newcommand{\diff}{\mathop{}\!\mathrm{d}}
\newcommand{\ie}{\emph{i.e. }}
%%%%%%%%%%%%%%%%%%%%%%%%%%%%%%%%%%%%%%%%%%%%%%%%%%%%%%


%% Beamer Layout %%%%%%%%%%%%%%%%%%%%%%%%%%%%%%%%%%
\useoutertheme[subsection=false,shadow]{miniframes}
\useinnertheme{default}
\usefonttheme{serif}
\usepackage{palatino}

\setbeamerfont{title like}{shape=\scshape}
\setbeamerfont{frametitle}{shape=\scshape}

\setbeamercolor*{lower separation line head}{bg=DeepSkyBlue4} 
% \setbeamercolor*{normal text}{fg=black,bg=white} 
% \setbeamercolor*{alerted text}{fg=red} 
% \setbeamercolor*{example text}{fg=black} 
% \setbeamercolor*{structure}{fg=black} 
%  
\setbeamercolor*{palette tertiary}{fg=black,bg=black!10} 
\setbeamercolor*{palette quaternary}{fg=black,bg=black!10} 
% 
% \renewcommand{\(}{\begin{columns}}
% \renewcommand{\)}{\end{columns}}
% \newcommand{\<}[1]{\begin{column}{#1}}
% \renewcommand{\>}{\end{column}}
%%%%%%%%%%%%%%%%%%%%%%%%%%%%%%%%%%%%%%%%%%%%%%%%%%




\begin{document}


%%%%%%%%%%%%%%%%%%%%%%%%%%%%%%%%%%%%%%%%%%%%%%%%%%%%%%
%%%%%%%%%%%%%%%%%%%%%%%%%%%%%%%%%%%%%%%%%%%%%%%%%%%%%%
\begin{frame}
\title{Génaration d'éléments normaux en C}
%\subtitle{SUBTITLE}
% \author{
% 	Édouard Rousseau\\
% 	{\it Université de Versailles}\\
% }
\date{\today}
\titlepage
\end{frame}

%%%%%%%%%%%%%%%%%%%%%%%%%%%%%%%%%%%%%%%%%%%%%%%%%%%%%%
%%%%%%%%%%%%%%%%%%%%%%%%%%%%%%%%%%%%%%%%%%%%%%%%%%%%%%
\begin{frame}{Table des matières}
\tableofcontents
\end{frame}

%%%%%%%%%%%%%%%%%%%%%%%%%%%%%%%%%%%%%%%%%%%%%%%%%%%%%%
%%%%%%%%%%%%%%%%%%%%%%%%%%%%%%%%%%%%%%%%%%%%%%%%%%%%%%
\section{\scshape Introduction}
\subsection{Contexte}
\begin{frame}{Contexte}
  On travail dans des \emph{extensions de corps finis} 
  \[
    \mathbb{F}_{p^d}/\mathbb{F}_p
  \]
  où $\mathbb{F}_p\cong \mathbb{Z}/p\mathbb{Z}$ est le corps à $p$ éléments, et
  plus généralement $\mathbb{F}_{p^d}\cong\mathbb{F}_p[X]/(P(X))$ (où $P$ est un
  polynôme irréductible de degré $d$ de $\mathbb{F}_p[X]$) est le corps à $p^d$ éléments.
\end{frame}

%%%%%%%%%%%%%%%%%%%%%%%%%%%%%%%%%%%%%%%%%%%%%%%%%%%%%%
%%%%%%%%%%%%%%%%%%%%%%%%%%%%%%%%%%%%%%%%%%%%%%%%%%%%%%
\begin{frame}{Rappels et notations}
  \begin{prop}
  Le corps $\mathbb{F}_{p^d}$ est un $\mathbb{F}_p$-espace vectoriel de
  dimension $d$ : une base est $\left\{ 1, X, \dots, X^{d-1} \right\}$.
  \end{prop}
  \begin{defi}[automorphisme de Frobenius]
    On note $\sigma$ \emph{l'automorphisme de Frobenius}
    \[
      \begin{array}{cccc}
        \sigma: & \mathbb{F}_{p^d} & \to & \mathbb{F}_{p^d} \\
        & x & \mapsto & x^p.
      \end{array}
    \]
  \end{defi}
\end{frame}
\subsection{Éléments normaux}
\begin{frame}{Éléments normaux : définition}
  \begin{defi}[élément normal]
    Soit $\alpha\in\mathbb{F}_{p^d}$, on dit que $\alpha$ est un \emph{élément
    normal} si $\left\{ \alpha, \sigma(\alpha), \dots, \sigma^{d-1}(\alpha)
  \right\} = \left\{ \alpha, \alpha^p, \dots, \alpha^{p^{d-1}} \right\}$ est une
  base de $\mathbb{F}_{p^d}$.
  \end{defi}
  Ce type de base est appelé \emph{base normale}.
\end{frame}

\begin{frame}{Existence d'éléments normaux}
  \begin{thm}
    Soit $\mathbb{F}_{p^d}/\mathbb{F}_p$ une extension de corps finie. Il existe
    un élément normal $\alpha$ pour cette extension.
  \end{thm}
  Le but du projet était donc de réussir à générer de tels éléments.
\end{frame}

%%%%%%%%%%%%%%%%%%%%%%%%%%%%%%%%%%%%%%%%%%%%%%%%%%%%%%
%%%%%%%%%%%%%%%%%%%%%%%%%%%%%%%%%%%%%%%%%%%%%%%%%%%%%%
\section{\scshape Génération d'éléments normaux}
\begin{frame}{Présentation des algorithmes}
  On va voir trois algorithmes.

  \begin{itemize}
    \item Algorithme randomisé $O((d+\log p)(d\log p)^2)$
    \item Algorithme de Lüneburg $O((d^{\alert{2}<2>}+\log p)(d\log p)^2)$
    \item Algorithme de Lenstra $O((d^{\alert{2}<2>}+\log p)(d\log p)^2)$
  \end{itemize}

\end{frame}

%%%%%%%%%%%%%%%%%%%%%%%%%%%%%%%%%%%%%%%%%%%%%%%%%%%%%%
%%%%%%%%%%%%%%%%%%%%%%%%%%%%%%%%%%%%%%%%%%%%%%%%%%%%%%
\subsection{Algorithme randomisé}
\begin{frame}{Une propostion utile}
  \begin{prop}
    Soit $\alpha\in\mathbb{F}_{p^d}$ et
    $P=\alpha^{p^{d-1}}X^{d-1}+\dots+\alpha^pX+\alpha.$
    L'élément $\alpha$ est normal si et seulement si $P$ et $X^d-1$ sont
    premiers entre eux.
  \end{prop}
  On en déduit un algorithme de test pour savoir si un élément est normal.
\end{frame}

\begin{frame}{Algorithme randomisé naïf}
  On déduit également un algorithme pour générer des éléments normaux.
  \begin{itemize}
    \item<2-> Prendre un élément $\alpha\in\mathbb{F}_{p^d}$ au hasard.
    \item<3-> Vérifier s'il est normal, en utilisant la proposition précédente. 
    \item<4-> S'il est normal, on le garde; sinon, on recommence.
  \end{itemize}

  \uncover<5>{Mais on peut faire mieux.}
\end{frame}

\begin{frame}{Algorithme randomisé élaboré}
  \begin{prop}
    Soit $f(X)$ un polynôme irréductible de degré $d$ de $\mathbb{F}_p[X]$ et
    $\alpha$ une racine de $f$. Soit 
    \[
      g(X) = \frac{f(X)}{(X-\alpha)f'(\alpha)}.
    \]
    Il y a alors au moins $p-d(d-1)$ éléments $u\in\mathbb{F}_p$ tels que $g(u)$
    soit un élément normal de $\mathbb{F}_{p^d}$.
  \end{prop}
  Ainsi, si $p>2d(d-1)$, on a au moins une chance sur deux que $g(u)$ soit un
  élément normal.
\end{frame}

\begin{frame}{Algorithme randomisé élaboré}
  On a ainsi un algorithme plus efficace.
  \begin{itemize}
    \item<2-> Prendre un élément $u\in\mathbb{F}_{p}$ au hasard.
    \item<3-> Vérifier si $g(u)$ est normal. 
    \item<4-> Si $g(u)$ est normal, on s'arrête; sinon, on recommence.
  \end{itemize}

  \uncover<5>{Mais pour garantir que cet algorithme fonctionne, il faut au moins que
  $p>d(d-1)$.}
\end{frame}

\subsection{Algorithmes déterministes}
\begin{frame}{Polynômes de $\sigma$-ordre}
  \begin{defi}[polynôme de $\sigma$-ordre]
      Soit $\theta\in\mathbb{F}_{p^d}$ un élément. Soit $k$ le plus petit entier
      positif tel que $\sigma^k(\theta)=\theta^{p^k}$ appartienne au sous-espace
      vectoriel engendré par $\left\{ \sigma^i(\theta)\;|\;0\leq i<k \right\}$.
      Si $\sigma^k(\theta)=\sum_{i=0}^{k-1}c_i\sigma^i(\theta)$, alors on
      définit le polynôme de $\sigma$-ordre de $\theta$ comme
      \[
        \Ord_{\theta}(X)=X^k-\sum_{i=0}^{k-1}c_iX^i.
      \]
  \end{defi}
\end{frame}

\begin{frame}{Polynômes de $\sigma$-ordre : propriétés}
  \begin{prop}
   On a les propriétés suivantes :
   \begin{itemize}
     \item L'élément $\theta$ est normal si et seulement si $\Ord_\theta=X^d-1$.
     \item Si $P(\sigma)\theta = 0$, alors $\Ord_\theta$ divise $P$.
     \item Soit $\alpha,\beta\in\mathbb{F}_{p^d}$, $\Ord_{\alpha+\beta}$ divise
       $\ppcm(\Ord_\alpha, \Ord_\beta)$.
     \item Si $\Ord_\alpha$ et $\Ord_\beta$ sont premiers entre eux,
       $\Ord_{\alpha+\beta}=\Ord_\alpha\Ord_\beta$.
   \end{itemize}
  \end{prop}
\end{frame}

\begin{frame}{Algorithme de Lüneburg}
  \begin{itemize}
    \item<1-> Générer $f_i=\Ord_{X^i}$, pour $0\leq i < d$. 
      \begin{itemize}
        \item<2-> On a $\ppcm(f_0, \dots, f_{d-1})=X^d-1$.
      \end{itemize}
    \item<3-> Écrire, pour chaque $i$
      \[
        f_i=\prod_{j=1}^rg_{j}^{e_{i,j}}
      \]
      où les $g_j$ sont premiers entre eux deux à deux.
    \item<4-> Trouver $i(j)$ tel que $e_{i(j),j}$ soit maximal parmi les
      $e_{i,j}$.
    \item<5-> Calculer $h_j=f_{i(j)}/g_j^{e_{i(j),j}}$ puis
      $\beta_j=h_j(\sigma)X^{i(j)}$
      \begin{itemize}
        \item<6-> On a $\Ord_{\beta_j}=g_j^{e_{i(j),j}}$.
      \end{itemize}
    \item<7-> Alors $\beta=\sum_{j=1}^r\beta_j$ est normal.
      \begin{itemize}
        \item<8-> En effet
          $\Ord_\beta=\prod_{j=1}^r\Ord_{b_j}=\prod_{j=1}^rg_j^{e_{i(j),j}}=X^d-1$.
      \end{itemize}
  \end{itemize}
\end{frame}

\begin{frame}{Algorithme de Lenstra}
  L'algorithme de Lenstra fonctionne de la façon suivante.
  \begin{itemize}
    \item On prend un élément $\theta\in\mathbb{F}_{p^d}$ quelconque et on
      calcule $\Ord_\theta$.
    \item<2-> Si $\Ord_\theta=X^d-1$ on s'arrête; sinon, on modifie $\theta$ pour
      que $\Ord_\theta$ monte de degré.
  \end{itemize}
  
  \uncover<3>{Comme $\Ord_\theta$ divise $X^d-1$, l'algorithme s'arrête et on a
    bien un élément normal à la fin. Les modifications qu'on fait à $\theta$
    sont liées à de l'algèbre linéaire : on doit résoudre des systèmes
  linéaires.}

\end{frame}

\section{\scshape Résultats}
\begin{frame}{Résultats}
  \begin{itemize}
    \item L'algorithme randomisé semble avoir une complexité indentique à la
      complexité théorique, les calculs avec $d\approx2^{10}$ et $p\approx
      2^{21}$ (complexité en $2^{35}$) sont faits en une minute.
    \item De manière surprenante l'algorithme randomisé naïf va aussi vite que
      l'algorithme élaboré, après étude : les algorithmes naïf et élaboré semblent trouver des
      éléments normaux du premier coup, ce qui est étrange.
    \item L'algorithme de Lüneburg est assez lent.
    \item L'algorithme de Lenstra est plus rapide que celui de Lüneburg.
  \end{itemize}
\end{frame}

\end{document}

\documentclass[a4paper,11pt]{article}
\usepackage[utf8]{inputenc}
\usepackage[T1]{fontenc}
\usepackage[english]{babel}
\usepackage{amsmath,amssymb,amsthm,amsopn}
\usepackage{mathrsfs}
\usepackage{graphicx}
\usepackage{hyperref}
%\usepackage{tikz}
%\usepackage{array}
%\usepackage[top=1cm,bottom=1cm]{geometry}
%\usepackage{listings}
%\usepackage{xcolor}

% fancy headers and footers

%\usepackage{fancyhdr}
%\pagestyle{fancy}
%\fancyhead[L]{BCPST 2 - Lycée Jacques Prévert}
%\fancyhead[R]{Rappels d'analyse}
%\pagenumbering{gobble} % no page numbering

% Création des labels Théorème, Lemme, etc...

\newtheoremstyle{break}%
{}{}%
{\itshape}{}%
{\bfseries}{}%  % Note that final punctuation is omitted.
{\newline}{}

\newtheoremstyle{sc}%
{}{}%
{}{}%
{\scshape}{}%  % Note that final punctuation is omitted.
{\newline}{}

\theoremstyle{break}
\newtheorem{thm}{Théorème}[section]
\newtheorem{lm}[thm]{Lemme}
\newtheorem{prop}[thm]{Proposition}
\newtheorem{cor}[thm]{Corollaire}

\theoremstyle{sc}
\newtheorem{exo}{Exercice}

\theoremstyle{definition}
\newtheorem{defi}[thm]{Définition}
\newtheorem{ex}[thm]{Exemple}

\theoremstyle{remark}
\newtheorem{rem}[thm]{Remarque}

% Raccourcis pour les opérateurs mathématiques (les espaces avant-après sont modifiés pour mieux rentrer dans les codes mathématiques usuels)
\DeclareMathOperator{\Ker}{Ker}
\DeclareMathOperator{\Id}{Id}
\DeclareMathOperator{\Img}{Im}
\DeclareMathOperator{\Card}{Card}
\DeclareMathOperator{\Vect}{Vect}


% Nouvelles commandes
\newcommand{\ps}[2]{\left\langle#1,#2\right\rangle}
\newcommand{\ent}[2]{[\![#1,#2]\!]}
\newcommand{\diff}{\mathop{}\!\mathrm{d}}
\newcommand{\ie}{\emph{i.e. }}

% opening
\title{Normal bases generation in C}
\author{Édouard \textsc{Rousseau}\\Supervised by Michaël \textsc{Quisquater}}



\begin{document}

\maketitle

\begin{abstract}
  This is the report of a C project about the generation of normal elements in
  finite fields. Consider a field extension $\mathbb{F}_{p^d}/\mathbb{F}_p$, we
  say that $\alpha\in\mathbb{F}_{p^d}$ is normal if $\left\{
  \alpha,\alpha^p,\alpha^{p^2},\cdots,\alpha^{p^{d-1}} \right\}$ is a basis of
  $\mathbb{F}_{p^d}$ over $\mathbb{F}_p$. We first give some theory to
  characterize normal elements, then we describe three algorithms to compute
  normal elements : a randomized algorithm, Lüneburg's algorithm, and
  Lenstra's algorithm. Finally, we give experimental results about our
  implementation. All this work is based on Gao's PhD thesis~\cite{Ga93}.

\end{abstract}

\tableofcontents

\clearpage

\section{Introduction}
\subsection{Normal bases}
\subsection{Recalls and notations}
In all the text, we will note $\mathbb{F}_n$ the field with $n$ elements. We
recall that $n$ must be a \emph{prime power} (\ie $n=p^d$ where $p$ is a prime
number and $d$ is a positive number). We have
$\mathbb{F}_{p^d}\cong\mathbb{F}_p[X]/(P)$, where $P$ is an irreducible
polynomial of degree $d$ in $\mathbb{F}_p[X]$, and this is the representation
that will be used in the C code. In the following, we will note
$X$ both for the indeterminate $X$ and for its image $\bar X$ in the
quotient $\mathbb{F}_p[X]/(P)$. $\mathbb{F}_{p^d}$ is a vector space over
$\mathbb{F}_p$, of dimension $d$, the basis $\left\{ 1, X, X^2, \dots, 
X^{d-1} \right\}$ will be called the \emph{polynomial basis}. We also recall
that $\mathbb{F}_{p^d}$ is a \emph{field extension} of $\mathbb{F}_p$ and the
characteristic of $\mathbb{F}_{p^d}$ is $p$. Lastly, we denote by $\sigma$ the
\emph{Frobenius map}, defined by $\sigma(x)=x^p$ for all
$x\in\mathbb{F}_{p^d}$. This map is a $\mathbb{F}_p$-automorphism of
$\mathbb{F}_{p^d}$ (\ie $\sigma$ is a field morphism, a bijection, and $\forall
y\in\mathbb{F}_p,\;\sigma(y)=y$), as a consequence, $\sigma$ is also a linear
map.

All these recalls being made, we can begin our journey. Note that we look only
at \emph{prime extensions}, (\ie extensions of type
$\mathbb{F}_{p^d}/\mathbb{F}_p$). This choice has been made because the theory
of normal elements in extensions of type $\mathbb{F}_{q^n}/\mathbb{F}_q$ is not
different (replacing $p$ by $q$ in the following demontrations would be
sufficient), but the implementation of the algorithms is simpler in the case of
prime extensions.

\section{Basics on normal bases}

In all this section, we set $p$ a prime number and $d$ a positive number. We
work with the extension $\mathbb{F}_{p^d}/\mathbb{F}_p$ and the Frobenius
morphism $\sigma$ defined above.

\begin{defi}[normal element, normal basis]
  Let $\alpha\in\mathbb{F}_{p^d}$, we say that $\alpha$ is a \emph{normal
  element} if $\left\{ \alpha, \sigma(\alpha),\dots,\sigma^{d-1}(\alpha)
  \right\}=\left\{ \alpha, \alpha^{p}, \dots, \alpha^{p^{d-1}} \right\}$ is a
  basis of $\mathbb{F}_{p^d}$. Such a basis is called a \emph{normal basis}. 
\end{defi}

In order to recognize a normal element $\alpha$, we can compute the dimension of the
linear span of $\left\{ \sigma^i(\alpha) \;|\; 0\leq i < k \right\}$. A way of
doing that is to construct the matrix $M$ which columns are the coordinates of the
elements $\sigma^i(\alpha)$ in the polynomial basis, and to check if $M$ is
non-singular. This can be done using Gauss algorithm. This method is not
efficient so we will work on other characterizations of normal elements.



\section{Computation of normal bases}

\section{Experimental results}

\clearpage
\bibliographystyle{unsrt}
\bibliography{biblio}
\end{document}
